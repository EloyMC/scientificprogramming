\documentclass[10pt]{beamer}

\title{Scientific Programming}

\subtitle{Team Testing}

\author{Albert, Bastiaan, Mourad, Max, Misha, Eloy, Brendan}

\date{\today}

\begin{document}
	\begin{frame}
	\titlepage
	\end{frame}
	% Misha
    \begin{frame}
        \frametitle{Software testing research and software engineering education}
        \textbf{Problemen:}
        \begin{itemize}
            \item Onderzoek naar software testing houdt niet de huidige ontwikkelingen binnen software ontwerp en ontwikkeling bij.
            \item Onderzoekers hebben veel theoritische kennis, maar vaak een gebrek aan relevante praktijkervaring.
        \end{itemize}
        
        \textbf{Suggesties:}
        \begin{itemize}
            \item Hands-on training voor studenten en de faculteit.
            \item Focus leggen op de volgende onderwerpen binnen software engineering:
            \begin{itemize}
                \item testen van embedded systems
                \item testen van eigenschappen naast functionaliteit, bijvoorbeeld
                performance, safety en security
                \item simulatie
                \item industrieel empirisch onderzoek
                \item tools die testing technieken implementeren en gemakkelijk zijn in gebruik
            \end{itemize}
        \end{itemize}
    \end{frame}

	% Bastiaan
	\begin{frame}
		\frametitle{Testing Software Using Swarm Intelligence: A Bee Colony Optimization Approach}
		\begin{itemize}
			\item Automatiseren
			\item Search Based Software Testing en Bee Colony Optimization
			\item Verkenners en werkers
			\item Intensificatie en diversificatie strategie
		\end{itemize}
	\end{frame}

	% Albert
	\begin{frame}
		\frametitle{Test-Driven Development as an Innovation Value Chain}
		\begin{itemize}
			\item Agile methodology
			\item Process:
				\begin{itemize}
					\item 1. Design
					\item 2. Write unit tests
					\item 3. Code
				\end{itemize}
			\item Value chain:
				\begin{itemize}
					\item 1. Planning
					\item 2. Conceptualization
					\item 3. Project system level
					\item 4. Detailed project
					\item 5. Tests and sophistication
					\item 6. Product preparation
				\end{itemize}
		\end{itemize}
	\end{frame}

	% Eloy
	\begin{frame}
		\frametitle{Software Testing: The State of the Practice}
		\begin{itemize}
			\item Unit Testing
			\item Code Coverage
			\item Travis CI
		\end{itemize}
	\end{frame}	

	% Mourad
	\begin{frame}
		\frametitle{Software testing in a Scientific Research Group}
		\begin{itemize}
			\item Analyseren 
			\item Handmatig testen.
			\item Communicatie
			\item Survey
			\item Van handmatig test naar geautomatiseerde unit tests
		\end{itemize} 
	\end{frame}

	% Max
	\begin{frame}
		\frametitle{The Challenge of Testing Scientific Software}
		\textbf{Risico's:}
		\begin{itemize}
			\item Theorie
			\begin{itemize}
				\item Cognitieve complexiteit
				\item Validatie testing
				\item Continue modellen
			\end{itemize}
			\item Code implementatie
			\begin{itemize}
				\item Correctheid
				\item Documentatie
				\item Verificatie testen
			\end{itemize}
			\item Gebruik
		\end{itemize}
	\end{frame}

	\begin{frame}
		\frametitle{Testing scientific software: A systematic literature review}
		\begin{itemize}
			\item Doel: het vinden van problemen, de voorgestelde oplossingen en onopgeloste problemen waar tegenaan wordt gelopen bij het testen van wetenschappelijke software.
			\item Methode: Een systematische literatuur review.
			\item resultaten:
			De problemen bij het testen van software valt in een van de volgende categorieën:
			- De intrinsieke karakteristieken van de software. 
			- problemen door culturele verschillen van wetenschappers. onderscheid maken tussen code en de methode die het implementeert bijvoorbeeld
		\end{itemize} 
	\end{frame}

\end{document}
